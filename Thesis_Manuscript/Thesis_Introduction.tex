\documentclass[]{article}
\usepackage{lmodern}
\usepackage{amssymb,amsmath}
\usepackage{ifxetex,ifluatex}
\usepackage{fixltx2e} % provides \textsubscript
\ifnum 0\ifxetex 1\fi\ifluatex 1\fi=0 % if pdftex
  \usepackage[T1]{fontenc}
  \usepackage[utf8]{inputenc}
\else % if luatex or xelatex
  \ifxetex
    \usepackage{mathspec}
  \else
    \usepackage{fontspec}
  \fi
  \defaultfontfeatures{Ligatures=TeX,Scale=MatchLowercase}
\fi
% use upquote if available, for straight quotes in verbatim environments
\IfFileExists{upquote.sty}{\usepackage{upquote}}{}
% use microtype if available
\IfFileExists{microtype.sty}{%
\usepackage{microtype}
\UseMicrotypeSet[protrusion]{basicmath} % disable protrusion for tt fonts
}{}
\usepackage[margin=1in]{geometry}
\usepackage{hyperref}
\hypersetup{unicode=true,
            pdfborder={0 0 0},
            breaklinks=true}
\urlstyle{same}  % don't use monospace font for urls
\usepackage{graphicx,grffile}
\makeatletter
\def\maxwidth{\ifdim\Gin@nat@width>\linewidth\linewidth\else\Gin@nat@width\fi}
\def\maxheight{\ifdim\Gin@nat@height>\textheight\textheight\else\Gin@nat@height\fi}
\makeatother
% Scale images if necessary, so that they will not overflow the page
% margins by default, and it is still possible to overwrite the defaults
% using explicit options in \includegraphics[width, height, ...]{}
\setkeys{Gin}{width=\maxwidth,height=\maxheight,keepaspectratio}
\IfFileExists{parskip.sty}{%
\usepackage{parskip}
}{% else
\setlength{\parindent}{0pt}
\setlength{\parskip}{6pt plus 2pt minus 1pt}
}
\setlength{\emergencystretch}{3em}  % prevent overfull lines
\providecommand{\tightlist}{%
  \setlength{\itemsep}{0pt}\setlength{\parskip}{0pt}}
\setcounter{secnumdepth}{0}
% Redefines (sub)paragraphs to behave more like sections
\ifx\paragraph\undefined\else
\let\oldparagraph\paragraph
\renewcommand{\paragraph}[1]{\oldparagraph{#1}\mbox{}}
\fi
\ifx\subparagraph\undefined\else
\let\oldsubparagraph\subparagraph
\renewcommand{\subparagraph}[1]{\oldsubparagraph{#1}\mbox{}}
\fi

%%% Use protect on footnotes to avoid problems with footnotes in titles
\let\rmarkdownfootnote\footnote%
\def\footnote{\protect\rmarkdownfootnote}

%%% Change title format to be more compact
\usepackage{titling}

% Create subtitle command for use in maketitle
\providecommand{\subtitle}[1]{
  \posttitle{
    \begin{center}\large#1\end{center}
    }
}

\setlength{\droptitle}{-2em}

  \title{}
    \pretitle{\vspace{\droptitle}}
  \posttitle{}
    \author{}
    \preauthor{}\postauthor{}
    \date{}
    \predate{}\postdate{}
  

\begin{document}

\hypertarget{introduction}{%
\section{Introduction}\label{introduction}}

The Olympia oyster (Ostrea lurida) is native to the Pacific North
American coast, ranging from British Columbia, Canada (Polson and
Zacherl 2009) to Baja California, Mexico (Raith et al.~2016).
Overharvest of O. lurida and anthropogenic pollution led to the
introduction of non-native oyster species for aquaculture to subsidize
the dwindling native oyster harvests starting in 1904 (reviewed in
(Baker 1995)). Currently, O. lurida abundances are less than 1\% of
historical estimates in California (Beck et al.~2011), while the
Japanese oyster (Crassostrea gigas) has established feral populations
outside of aquaculture operations in Southern California (Crooks, Crooks
and Crooks 2015; Zacherl, personal observations). The extreme reduction
in native O. lurida abundances is coupled with a loss of important hard
substrate habitat and its associated ecosystem services. The ecosystem
service of interest to this research is water filtration. Oysters are
suspension feeders that remove phytoplankton, bacteria, algae, and
suspended sediments from the water column, increasing water clarity
(Newell and Koch 2004; Grizzle, Greene and Coen 2008) useful to benthic
primary producers, like seagrasses (Peterson and Heck 1999).

zu Ermgassen et al. (2016) provides a management tool to calculate
filtration services provided by O. lurida habitat. However, O. lurida
filtration is calculated using data from only one laboratory study (zu
Ermgassen et al., 2013), while the eastern oyster (Crassostrea
virginica) filtration model uses a combination of laboratory and in situ
data. In situ filtration rates of O. lurida habitat are important to
estimate filtration services under complex environmental conditions
often difficult to simulate in a laboratory setting. Gray and Langdon
(2017) measured in situ filtration of isolated O. lurida in Yaquina Bay,
Oregon, providing a foundation for estimating in situ O. lurida
filtration rates. However, three factors important to filtration remain
underexplored.

First, O. lurida beds are complex habitat used by other filter feeding
organisms (i.e.~mussels, tunicates, scallops, and sponges) that may
substantially contribute to net filtration, and differ among bays.
Second, environmental conditions like temperature (Gray and Langdon
2017), seston organic content (Kreeger and Newell 2001; Gray and Langdon
2017), salinity (Gray and Langdon 2017), and water velocity (Grizzle,
Langan and Howell 1992; Luckenbach and Harsh 1999) are all known to
affect bivalve filtration rates, and are unique to each water body.
Third, research in Washington state showed that seston and phytoplankton
were regulated top-down by commercially grown C. gigas (Wheat and
Ruesink, 2013), which is a more robust filter feeder than O. lurida
(Bougrier et al.~1995; zu Ermgassen et al.~2013; Gray and Langdon 2017)
thus cultivated C. gigas may play a substantial role in overall water
filtration in Pacific bays. This and many other studies use chlorophyll
 as a metric for filtration (Luckenbach and Harsh 1999; Wasson et
al.~2015; Grizzle, Greene, and Coen 2008; Grizzle et al.~2006) because
chlorophyll  is a plant pigment whose concentration is a proxy for
phytoplankton and algae fragments consumed by filter feeders. In
summary, filtration models of oyster habitat lack in situ comparisons
that incorporate complex environmental conditions and other bivalve
contributions to overall filtration function of restored O. lurida
habitat and C. gigas aquaculture. My research uses in situ methods based
on Grizzle et al. (2006 \& 2008) and Milbrandt et al. (2015) to test
previous filtration models, using restored and aquaculture oyster
habitat.


\end{document}
